\documentclass{article}
%\usepackage{tex4ht}
\usepackage{amsmath}
\usepackage{amsfonts}
\usepackage{amssymb}
\usepackage{amsthm}
\usepackage{graphicx}
\usepackage{subfigure}
\usepackage{color}
\usepackage{float}

\setlength{\parskip}{0.5\baselineskip}
\title{Lecture Notes on Machine Learning}
\date{February 27, 2014}

\definecolor{lightgray}{gray}{0.5}
\newtheorem{definition}{Definition}
\newtheorem{lemma}{Lemma}
\newtheorem{theorem}{Theorem}
\newtheorem{propostion}{Propostion}
\newtheorem*{remark}{Remark}

\begin{document}
\maketitle

We will first introduce how to apply kernel trick on SVM. Suppose we map a set of observations $x_{i} \in \mathbb{R}^{p}$ to $X_{i} = \phi(x_{i})$,
where $X_{i}$ are in a Hilbert space $H$. Then, kernel SVM is to find a seperator with maximum margin in $H$:
\begin{equation*}
 G(X) = <U,X>_{H} + b
\end{equation*}
Let $\{X_{i}\}$ be the training set mapped into Hilbert space $H$, and $\{y_{i}\}$ be the corresponding label set. SVM requires that:
\begin{equation*}
 \begin{cases}
  G(X_{i}) \geq 1, \ when \ y_{i}=1 \\
  G(X_{i}) \leq -1, \ when \ y_{i}=-1
   \end{cases}
\end{equation*}
Introduce $\xi_{i}=\max(0, 1-y_{i}G(X_{i}))$, then we need to solve the following minimization problem:
\begin{equation*}
 \min\{\frac{1}{2}||U||^{2} + c\sum \xi_{i}\}
\end{equation*}
with constraints
\begin{equation*}
 \begin{cases}
  \xi_{i} \geq 0 \\
  \xi_{i}-(1-y_{i}G(x_{i})) \geq 0
 \end{cases}
\end{equation*}
There is an important theorem saying that the seperator $U$ will be the linear combination of $X_{i}$, i.e. $U=\sum \lambda_{i}X_{i}$, therefore, our
seperator can be written as:
\begin{equation*}
 L(x) = \sum \lambda_{i} k(x_{i},x) + b 
\end{equation*}
which is obviously a nonlinear classifier. Support vectors are those $x_{j}$ with $\lambda_{i}\neq0$.

Next, we move to the topic ``k-fold cross-validation''. It works as follows: given a data set, we seperate it into $k$ equal parts randomly. Test
on $i$th part and train on the others, we can do this $k$ times. Each time, we can get the good proportion $P_{i}^{+}$, $P_{i}^{-}$, which are the
proportions of correct classification into $+$ and $-$ classes. Then find the average $P^{+}$, $P^{-}$ to evalute the performance of the classify
method.

One of the most famous cross-validation method is ``leave one-out'', which is: if the data set is small, each time we take one observation out as
test set, and the others compose training set to build classifier. We can do this many times (same as the number of observations), and take the
average of good proportions to evalute the performance.

(For the example on biomedical data, I really did not understand well, so I ignore this part...)

\end{document}


